% !TeX encoding = UTF-8
% !TeX program = LuaLaTeX
% !TeX spellcheck = en_US

% Author : pppppass
% Description : Outline: LaTeX --- Seminar on Selected Tools Week 0 --- Python, LaTeX and Git

\documentclass[english, nochinese]{../TeXTemplate/pkuslide}

\usepackage{../TeXTemplate/def}

\title{Outline: \texorpdfstring{\LaTeX}{LaTeX}}
\subtitle{Seminar on Selected Tools Week 0 --- Python, \texorpdfstring{\LaTeX}{LaTeX} and Git}

\author{pppppass}
\date{Updated on January 23, 2018}

\subject{Outline: \texorpdfstring{\LaTeX}{LaTeX} --- Seminar on Selected Tools Week 0 --- Python, \texorpdfstring{\LaTeX}{LaTeX} and Git}
\keywords{}

\begin{document}

\begin{frame}
\titlepage
\end{frame}

\begin{frame}
\tableofcontents[subsectionstyle=show]
\end{frame}

\section{Introduction}

\begin{frame}
\sectionpage
\end{frame}

\subsection{\TeX{} and \LaTeX}

\begin{frame}{What is \TeX{} and \LaTeX?}
\begin{enumerate}
\item \TeX{} is a typesetting system originally designed by Donald Knuth.
\item \LaTeX{} is a typesetting system based on \TeX{} designed \\
originally designed by Lesile Lamport.
\item Basic idea of \TeX{} and \LaTeX{}: ``What you think is what you get'', \\ distinguished from ``What you see is what you get''
\item Programming mechanism of \TeX{} and \LaTeX{} is based on macros.
\end{enumerate}
\end{frame}

\begin{frame}{Why to use \LaTeX?}
\begin{enumerate}
\item Beautiful and elegant layout and fonts
\item Full and explicit control of details
\item Very easy to handle structured materials \\
e.g. papers, books, notes
\item De facto standard for mathematics, physics and computer science
\item Widely used math modes for mathematical formulas \\
e.g. Markdown, websites and even daily communication
\item A great number of packages, and an activate community
\end{enumerate}

When not to use \LaTeX{}: the material is highly unstructured
\end{frame}

\subsection{Basic concepts}

\begin{frame}{Basic concepts}
\begin{enumerate}
\item \TeX{} engines and \TeX{} formats
\item Commands (control sequences) and arguments
\item Whitespaces
\item Envorinments and groups
\item Math modes: inline and displayed
\item Packages and document classes
\item Floats: figures and tables
\item Bibliography tools
\end{enumerate}
\end{frame}

\begin{frame}[fragile]{Basic structure of a document}
\begin{enumerate}
\item Command \verb"\documentclass{...}":
\verb"article", \verb"ctexart" and \verb"beamer"
\item Preamble: definitions and \verb"\usepackage{...}"s
\item Begin a document environment: \verb"\begin{document}"
\item Top matters: \verb"\title{...}", \verb"\author{...}" and \verb"\date{...}"
\item Section: \verb"\section{...}" and so on
\item Paragraphs separated by a single blank line
\item End a document environment: \verb"\end{document}"
\end{enumerate}
\end{frame}

\section{Basic typesetting}

\begin{frame}
\sectionpage
\end{frame}

\subsection{Text mode}

\begin{frame}[fragile]{Text formatting}
\begin{enumerate}
\item Special characters: \verb"\&", \verb"\_", \verb"\S"
\item Whitespace: \verb"\ ", \verb"\!", \verb"\phantom{...}" and \verb"\hspace{...}"
Use two blank lines to initiate a new paragraph
\item Paragraphs: \verb"\\", \verb"\par" and two blank lines
\item Orthogonal coordinates of fonts: \\
English: \verb"\ttfamily", \verb"\bfseries", \verb"\textrm{...}", \verb"\textit{...}" \\
Chinese: \verb"\kaishu", \verb"\heiti"
\item Emphasize: \verb"\emph{...}"
\item Font size: \verb"\tiny", \verb"\small", \verb"\large", \verb"\Large", \verb"\LARGE"
\item Align: \verb"\centering", \verb"\raggedright"
\end{enumerate}
\end{frame}

\begin{frame}[fragile]{Basic environments}
\begin{enumerate}
\item Quotes: \verb"quote"
\item Lists: \verb"enumerate", \verb"itemize" and package \verb"enumitem"
\item Theorems: \verb"\newtheorem" and package \verb"ntheorem"
\item Verbatim: \verb"\verb'...'"
\item Program lists: \verb"lstlisting" and package \verb"listings"
\end{enumerate}
\end{frame}

\subsection{Math mode}

\begin{frame}[fragile]{Formula formatting}
\begin{enumerate}
\item Use \verb"$" for inline formulas \\
and envorinment \verb"equation" for displayed ones
\item Symbols
\item Formula structures
\item Environments
\item Package \verb"amsmath"
\end{enumerate}
\end{frame}

\begin{frame}[fragile]{Mathematical symbols}
\begin{enumerate}
\item Types: normal texts, operators, binary operators, relations accents
\item Fonts: \verb"\mathrm{...}", \verb"\mathbf{...}"
\item Normal symbols: \verb"\exists", \verb"\forall"
\item Operators: \verb"\log", \verb"\sin"
\item Binary operators: \verb"+", \verb"\setminus", \verb"\otimes"
\item Relations: \verb"\le", \verb"\equiv", \verb"\approx"
\item Whitespace: \verb"\,", \verb"\!" \\
frequently used: \verb"\mathop{\mathrm{d}\!} x"
\end{enumerate}
\end{frame}

\begin{frame}[fragile]{Formula structures}
\begin{enumerate}
\item Subscript and superscript: \verb"_" and \verb"^"
\item Fraction and binominals: \verb"\frac{...}{...}", \verb"\binom{...}{...}"
\item Roots and radicals: \verb"\sqrt{...}"
\item Huge operators: \verb"\sum", \verb"\product", \verb"\bigoplus"
\item Delimiters: \verb"\left", \verb"\right", and brakets like \verb"\lfloor"
\item Matrices: environment \verb"matrix", \verb"bmatrix" and package \verb"amsmath"
\end{enumerate}
\end{frame}

\begin{frame}[fragile]{Mathematical environments}
\begin{enumerate}
\item Basic equations: enviromnet \verb"equation"
\item Matrices: environment \verb"matrix", \verb"bmatrix"
\item If-cases: environment \verb"cases"
\item Gathered equations: environment \verb"gather"
\item Aligned equations: environment \verb"align"
\item Formulas in formulas: environment \verb"split", \verb"gathered" and \verb"aligned"
\end{enumerate}
\end{frame}

\section{Further topics}

\begin{frame}
\sectionpage
\end{frame}

\begin{frame}[fragile]{Bibliography}
\begin{enumerate}
\item BibTeX
\item Footnote: \verb"\footnote{...}"
\item Cite: \verb"\cite{...}"
\item Display: \verb"\printbibliography"
\item BibLaTeX is also available
\end{enumerate}
\end{frame}

\begin{frame}[fragile]{Floats}
\begin{enumerate}
\item Environment \verb"figure"
\item Environment \verb"table"
\item Package \verb"graphicx"
\end{enumerate}
\end{frame}

\begin{frame}[fragile]{Tables}
\begin{enumerate}
\item Environment \verb"tabular" and \verb"array"
\item Column formats: e.g. \verb"|c|rrrlr|"
\item Align: \verb"&" and \verb"\\"
\item Row lines: \verb"\hline"
\end{enumerate}
\end{frame}

\begin{frame}[fragile]{Other useful packages}
\begin{enumerate}
\item Layout: \verb"geometry"
\item Longer table: \verb"longtable"
\item Multiple integrals: \verb"esint"
\item Calligraphy: \verb"mathrsfs"
\item Algorithms: \verb"algorithm2e" or \verb"algorithm"
\item Hyper-links: \verb"hyperref"
\item Include .pdf files: \verb"pdfpages"
\item Thesis: \verb"pkuthss", \verb"thuthesis"
\end{enumerate}
\end{frame}

\end{document}
