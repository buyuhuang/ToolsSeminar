% !TeX encoding = UTF-8
% !TeX program = LuaLaTeX
% !TeX spellcheck = en_US

% Author : pppppass
% Description : Python Assignment --- K-Means

\documentclass[english, nochinese]{../../TeXTemplate/pkupaper}

\usepackage[paper, listings]{../../TeXTemplate/def}

\newcommand{\cuniversity}{}
\newcommand{\cthesisname}{Python Assignment: \texorpdfstring{$K$}{K}-Means}
\newcommand{\titlemark}{Python Assignment: \texorpdfstring{$K$}{K}-Means}

\title{\titlemark}
\author{pppppass}
\date{January 22, 2018}

\begin{document}

\maketitle

\section{Description}

Implement $K$-Means algorithm to a specific $2$-dimensional dataset.

\section{Requirement}

Implement a function \verb"k_means" in \verb"k_means.py". The function should accept \verb"k" as the value of $k$, \verb"data" as data, \verb"cid" as the index of initial center points of the two clusters, and \verb"rep" as the number of iterations. The argument \verb"data" is a list of $2$-element lists representing coordinates of points, and \verb"cid" is given as a list of integers.

The file \verb"interface.py" contains the interface to be tested.

\section{Usage of tools}

You may use Make to perform automatic test. Command \verb"make run" runs the script with \verb"k_means.py", and command \verb"make key" runs the script with \verb"key.py" substituting \verb"k_means.py", giving the desired result. Command \verb"make test" tests your \verb"k_means.py".

\end{document}
